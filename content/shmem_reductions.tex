\apisummary{
    Performs arithmetic and logical operations across a set of \acp{PE}.
}

\begin{apidefinition}

\textbf{AND} \newline
Performs a bitwise AND function across a set of processing elements (\acp{PE}).\newline
\begin{Csynopsis}
void shmem_short_and_to_all(short *dest, const short *source, int nreduce, int PE_start, int logPE_stride, int PE_size, short *pWrk, long *pSync);
void shmem_int_and_to_all(int *dest, const int *source, int nreduce, int PE_start, int logPE_stride, int PE_size, int *pWrk, long *pSync);
void shmem_long_and_to_all(long *dest, const long *source, int nreduce, int PE_start, int logPE_stride, int PE_size, long *pWrk, long *pSync);
void shmem_longlong_and_to_all(long long *dest, const long long *source, int nreduce, int PE_start, int logPE_stride, int PE_size, long long *pWrk, long *pSync);
\end{Csynopsis}

\begin{Fsynopsis}
CALL SHMEM_INT4_AND_TO_ALL(dest, source, nreduce, PE_start, logPE_stride, PE_size, pWrk, pSync)
CALL SHMEM_INT8_AND_TO_ALL(dest, source, nreduce, PE_start, logPE_stride, PE_size, pWrk, pSync)
\end{Fsynopsis}

\bigskip
\textbf{MAX} \newline
Performs a maximum function reduction across a set of processing elements (\acp{PE}).\newline
\begin{Csynopsis}
void shmem_short_max_to_all(short *dest, const short *source, int nreduce, int PE_start, int logPE_stride, int PE_size, short *pWrk, long *pSync);
void shmem_int_max_to_all(int *dest, const int *source, int nreduce, int PE_start, int logPE_stride, int PE_size, int *pWrk, long *pSync);
void shmem_double_max_to_all(double *dest, const double *source, int nreduce, int PE_start, int logPE_stride, int PE_size, double *pWrk, long *pSync);
void shmem_float_max_to_all(float *dest, const float *source, int nreduce, int PE_start, int logPE_stride, int PE_size, float *pWrk, long *pSync);
void shmem_long_max_to_all(long *dest, const long *source, int nreduce, int PE_start, int logPE_stride, int PE_size, long *pWrk, long *pSync);
void shmem_longdouble_max_to_all(long double *dest, const long double *source, int nreduce, int PE_start, int logPE_stride, int PE_size, long double *pWrk, long *pSync);
void shmem_longlong_max_to_all(long long *dest, const long long *source, int nreduce, int PE_start, int logPE_stride, int PE_size, long long *pWrk, long *pSync);
\end{Csynopsis}

\begin{Fsynopsis}
CALL SHMEM_INT4_MAX_TO_ALL(dest, source, nreduce, PE_start, logPE_stride, PE_size, pWrk, pSync)
CALL SHMEM_INT8_MAX_TO_ALL(dest, source, nreduce, PE_start, logPE_stride, PE_size, pWrk, pSync)
CALL SHMEM_REAL4_MAX_TO_ALL(dest, source, nreduce, PE_start, logPE_stride, PE_size, pWrk, pSync)
CALL SHMEM_REAL8_MAX_TO_ALL(dest, source, nreduce, PE_start, logPE_stride, PE_size, pWrk, pSync)
CALL SHMEM_REAL16_MAX_TO_ALL(dest, source, nreduce, PE_start, logPE_stride, PE_size, pWrk, pSync)
\end{Fsynopsis}

\bigskip
\textbf{MIN} \newline
Performs a minimum function reduction across a set of processing elements (\acp{PE}).\newline
\begin{Csynopsis}
void shmem_short_min_to_all(short *dest, const short *source, int nreduce, int PE_start, int logPE_stride, int PE_size, short *pWrk, long *pSync);
void shmem_int_min_to_all(int *dest, const int *source, int nreduce, int PE_start, int logPE_stride, int PE_size, int *pWrk, long *pSync);
void shmem_double_min_to_all(double *dest, const double *source, int nreduce, int PE_start, int logPE_stride, int PE_size, double *pWrk, long *pSync);
void shmem_float_min_to_all(float *dest, const float *source, int nreduce, int PE_start, int logPE_stride, int PE_size, float *pWrk, long *pSync);
void shmem_long_min_to_all(long *dest, const long *source, int nreduce, int PE_start, int logPE_stride, int PE_size, long *pWrk, long *pSync);
void shmem_longdouble_min_to_all(long double *dest, const long double *source, int nreduce, int PE_start, int logPE_stride, int PE_size, long double *pWrk, long *pSync);
void shmem_longlong_min_to_all(long long *dest, const long long *source, int nreduce, int PE_start, int logPE_stride, int PE_size, long long *pWrk, long *pSync);
\end{Csynopsis}

\begin{Fsynopsis}
CALL SHMEM_INT4_MIN_TO_ALL(dest, source, nreduce, PE_start, logPE_stride, PE_size, pWrk, pSync)
CALL SHMEM_INT8_MIN_TO_ALL(dest, source, nreduce, PE_start, logPE_stride, PE_size, pWrk, pSync)
CALL SHMEM_REAL4_MIN_TO_ALL(dest, source, nreduce, PE_start, logPE_stride, PE_size, pWrk, pSync)
CALL SHMEM_REAL8_MIN_TO_ALL(dest, source, nreduce, PE_start, logPE_stride, PE_size, pWrk, pSync)
CALL SHMEM_REAL16_MIN_TO_ALL(dest, source, nreduce, PE_start, logPE_stride, PE_size, pWrk, pSync)
\end{Fsynopsis}

\bigskip
\textbf{SUM} \newline
Performs a sum reduction across a set of processing elements (\acp{PE}).\newline
\begin{Csynopsis}
void shmem_complexd_sum_to_all(double complex *dest, const double complex *source, int nreduce, int PE_start, int logPE_stride, int PE_size, double complex *pWrk, long |\mbox{*pSync);}|
void shmem_complexf_sum_to_all(float complex *dest, const float complex *source, int nreduce, int PE_start, int logPE_stride, int PE_size, float complex *pWrk, long *pSync);
void shmem_short_sum_to_all(short *dest, const short *source, int nreduce, int PE_start, int logPE_stride, int PE_size, short *pWrk, long *pSync);
void shmem_int_sum_to_all(int *dest, const int *source, int nreduce, int PE_start, int logPE_stride, int PE_size, int *pWrk, long *pSync);
void shmem_double_sum_to_all(double *dest, const double *source, int nreduce, int PE_start, int logPE_stride, int PE_size, double *pWrk, long *pSync);
void shmem_float_sum_to_all(float *dest, const float *source, int nreduce, int PE_start, int logPE_stride, int PE_size, float *pWrk, long *pSync);
void shmem_long_sum_to_all(long *dest, const long *source, int nreduce, int PE_start, int logPE_stride,int PE_size, long *pWrk, long *pSync);
void shmem_longdouble_sum_to_all(long double *dest, const long double *source, int nreduce, int PE_start, int logPE_stride, int PE_size, long double *pWrk, long *pSync);
void shmem_longlong_sum_to_all(long long *dest, const long long *source, int nreduce, int PE_start, int logPE_stride, int PE_size, long long *pWrk, long *pSync);
\end{Csynopsis}

\begin{Fsynopsis}
CALL SHMEM_COMP4_SUM_TO_ALL(dest, source, nreduce, PE_start, logPE_stride, PE_size, pWrk, pSync)
CALL SHMEM_COMP8_SUM_TO_ALL(dest, source, nreduce, PE_start, logPE_stride, PE_size, pWrk, pSync)
CALL SHMEM_INT4_SUM_TO_ALL(dest, source, nreduce, PE_start, logPE_stride, PE_size, pWrk, pSync)
CALL SHMEM_INT8_SUM_TO_ALL(dest, source, nreduce, PE_start, logPE_stride, PE_size, pWrk, pSync)
CALL SHMEM_REAL4_SUM_TO_ALL(dest, source, nreduce, PE_start, logPE_stride, PE_size, pWrk, pSync)
CALL SHMEM_REAL8_SUM_TO_ALL(dest, source, nreduce, PE_start, logPE_stride, PE_size, pWrk, pSync)
CALL SHMEM_REAL16_SUM_TO_ALL(dest, source, nreduce, PE_start, logPE_stride, PE_size, pWrk, pSync)
\end{Fsynopsis}

\bigskip
\textbf{PROD} \newline
Performs a product reduction across a set of processing elements (\acp{PE}).\newline
\begin{Csynopsis}
void shmem_complexd_prod_to_all(double complex *dest, const double complex *source, int nreduce, int PE_start, int logPE_stride, int PE_size, double complex *pWrk, long |\mbox{*pSync);}|
void shmem_complexf_prod_to_all(float complex *dest, const float complex *source, int |\mbox{nreduce,}| int PE_start, int logPE_stride, int PE_size, float complex *pWrk, long *pSync);
void shmem_short_prod_to_all(short *dest, const short *source, int nreduce, int PE_start, int logPE_stride, int PE_size, short *pWrk, long *pSync);
void shmem_int_prod_to_all(int *dest, const int *source, int nreduce, int PE_start, int logPE_stride, int PE_size, int *pWrk, long *pSync);
void shmem_double_prod_to_all(double *dest, const double *source, int nreduce, int PE_start, int logPE_stride, int PE_size, double *pWrk, long *pSync);
void shmem_float_prod_to_all(float *dest, const float *source, int nreduce, int PE_start, int logPE_stride, int PE_size, float *pWrk, long *pSync);
void shmem_long_prod_to_all(long *dest, const long *source, int nreduce, int PE_start, int logPE_stride, int PE_size, long *pWrk, long *pSync);
void shmem_longdouble_prod_to_all(long double *dest, const long double *source, int nreduce, int PE_start, int logPE_stride, int PE_size, long double *pWrk, long *pSync);
void shmem_longlong_prod_to_all(long long *dest, const long long *source, int nreduce, int PE_start, int logPE_stride, int PE_size, long long *pWrk, long *pSync);
\end{Csynopsis}

\begin{Fsynopsis}
CALL SHMEM_COMP4_PROD_TO_ALL(dest, source, nreduce, PE_start, logPE_stride, PE_size, pWrk, pSync)
CALL SHMEM_COMP8_PROD_TO_ALL(dest, source, nreduce, PE_start, logPE_stride, PE_size, pWrk, pSync)
CALL SHMEM_INT4_PROD_TO_ALL(dest, source, nreduce, PE_start, logPE_stride, PE_size, pWrk, pSync)
CALL SHMEM_INT8_PROD_TO_ALL(dest, source, nreduce, PE_start, logPE_stride, PE_size, pWrk, pSync)
CALL SHMEM_REAL4_PROD_TO_ALL(dest, source, nreduce, PE_start, logPE_stride, PE_size, pWrk, pSync)
CALL SHMEM_REAL8_PROD_TO_ALL(dest, source, nreduce, PE_start, logPE_stride, PE_size, pWrk, pSync)
CALL SHMEM_REAL16_PROD_TO_ALL(dest, source, nreduce, PE_start, logPE_stride, PE_size, pWrk, pSync)
\end{Fsynopsis}

\bigskip
\textbf{OR} \newline
Performs  a  bitwise  OR  function reduction across a set of processing elements (\acp{PE}).\newline
\begin{Csynopsis}
void shmem_short_or_to_all(short *dest, const short *source, int nreduce, int PE_start, int logPE_stride, int PE_size, short *pWrk, long *pSync);
void shmem_int_or_to_all(int *dest, const int *source, int nreduce, int PE_start, int logPE_stride, int PE_size, int *pWrk, long *pSync);
void shmem_long_or_to_all(long *dest, const long *source, int nreduce, int PE_start, int logPE_stride, int PE_size, long *pWrk, long *pSync);
void shmem_longlong_or_to_all(long long *dest, const long long *source, int nreduce, int PE_start, int logPE_stride, int PE_size, long long *pWrk, long *pSync);
\end{Csynopsis}

\begin{Fsynopsis}
CALL SHMEM_INT4_OR_TO_ALL(dest, source, nreduce, PE_start, logPE_stride, PE_size, pWrk, |\mbox{pSync)}|
CALL SHMEM_INT8_OR_TO_ALL(dest, source, nreduce, PE_start, logPE_stride, PE_size, pWrk, |\mbox{pSync)}|	
\end{Fsynopsis}

\bigskip
\textbf{XOR}\newline
Performs  a  bitwise  EXCLUSIVE OR reduction across a set of processing elements (\acp{PE}).\newline
\begin{Csynopsis}
void shmem_short_xor_to_all(short *dest, const short *source, int nreduce, int PE_start, int logPE_stride, int PE_size, short *pWrk, long *pSync);
void shmem_int_xor_to_all(int *dest, const int *source, int nreduce, int PE_start, int logPE_stride, int PE_size, int *pWrk, long *pSync);
void shmem_long_xor_to_all(long *dest, const long *source, int nreduce, int PE_start, int logPE_stride, int PE_size, long *pWrk, long *pSync);
void shmem_longlong_xor_to_all(long long *dest, const long long *source, int nreduce, int PE_start, int logPE_stride, int PE_size, long long *pWrk, long *pSync);
\end{Csynopsis}

\begin{Fsynopsis}
CALL SHMEM_INT4_XOR_TO_ALL(dest, source, nreduce, PE_start, logPE_stride, PE_size, pWrk, pSync)
CALL SHMEM_INT8_XOR_TO_ALL(dest, source, nreduce, PE_start, logPE_stride, PE_size, pWrk, pSync)
\end{Fsynopsis}

\begin{apiarguments}

\apiargument{IN}{dest}{A symmetric array, of length \VAR{nreduce} elements, to
    receive the result of the reduction routines.  The data type of \dest{} varies
    with the version of the reduction routine being called.  When calling from
    \CorCpp, refer to the SYNOPSIS section for data type information.}
\apiargument{IN}{source}{ A symmetric array, of length \VAR{nreduce} elements, that
    contains one element for each separate reduction routine.  The \source{}
    argument must have the same data type as \dest.}
\apiargument{IN}{\VAR{nreduce}}{The number of elements in the \dest{} and \source{}
    arrays.  \VAR{nreduce} must be of type integer.  When using \Fortran, it
    must be a default integer value.}
\apiargument{IN}{PE\_start}{The lowest \ac{PE} number of the \activeset{} of
    \acp{PE}.  \VAR{PE\_start} must be of type integer.  When using \Fortran,
    it must be a default integer value.}
\apiargument{IN}{logPE\_stride}{The log (base 2) of the stride between consecutive
    \ac{PE} numbers in the \activeset.  \VAR{logPE\_stride} must be of type integer.
    When using \Fortran, it must be a default integer value.}
\apiargument{IN}{PE\_size}{The number of \acp{PE} in the \activeset.
    \VAR{PE\_size} must be of type integer.  When using \Fortran, it must be a
    default integer value.}
\apiargument{IN}{pWrk}{A symmetric work array. The \VAR{pWrk} argument must have the
    same data type as \dest. In \CorCpp, this contains max(\VAR{nreduce}/2 + 1,
    \CONST{SHMEM\_REDUCE\_MIN\_WRKDATA\_SIZE}) elements. In \Fortran, this
    contains max(\VAR{nreduce}/2 + 1, \CONST{SHMEM\_REDUCE\_MIN\_WRKDATA\_SIZE})
    elements.}
\apiargument{IN}{pSync}{A symmetric work array. In \CorCpp, \VAR{pSync} must be of
    type long and size \CONST{SHMEM\_REDUCE\_SYNC\_SIZE}. In \Fortran, \VAR{pSync}
    must be of type integer and size \CONST{SHMEM\_REDUCE\_SYNC\_SIZE}.  When
    using \Fortran, it must be a default integer value. Every element of this array
    must be initialized with the value \CONST{SHMEM\_SYNC\_VALUE} (in \CorCpp) or
    \CONST{SHMEM\_SYNC\_VALUE} (in \Fortran) before any of the \acp{PE} in the
    \activeset{} enter the reduction routine.}
    
\end{apiarguments}

\apidescription{
    \openshmem reduction routines compute one or more reductions across symmetric
    arrays on multiple \acp{PE}.  A reduction performs an associative binary routine
    across a set of values.	 
    
    The \VAR{nreduce} argument determines the number of separate reductions to
    perform.  The \source{} array on all \acp{PE} in the \activeset{} provides one
    element for each reduction.  The results of the reductions are placed in the
    \dest{} array on all \acp{PE} in the \activeset.  The \activeset{} is defined
    by the \VAR{PE\_start}, \VAR{logPE\_stride}, \VAR{PE\_size} triplet.
    
    The \source{} and \dest{} arrays may be the same array, but they may not be
    overlapping arrays.
    
    As with all \openshmem{} collective routines, each of these routines assumes
    that only \acp{PE} in the \activeset{} call the routine.  If a \ac{PE} not in
    the \activeset{} calls an \openshmem collective routine, undefined behavior
    results.
    
    The values of arguments \VAR{nreduce}, \VAR{PE\_start}, \VAR{logPE\_stride}, and
    \VAR{PE\_size} must be equal on all \acp{PE} in the \activeset. The same \dest{}
    and \source{} arrays, and the same \VAR{pWrk} and \VAR{pSync} work arrays, must
    be passed to all \acp{PE} in the \activeset.
    %FIXME: Reword 'the following conditions must be met.'
    Before any \ac{PE} calls a reduction routine, the
    following conditions must be met (synchronization via a \OPR{barrier} or some other
    method is often needed to ensure this): The \VAR{pWrk} and \VAR{pSync} arrays
    on all \acp{PE} in the \activeset{} are not still in use from a prior call to a
    collective \openshmem{} routine.  The \dest{} array on all \acp{PE} in the
    \activeset{} is ready to accept the results of the \OPR{reduction}.
    
    Upon return from a reduction routine, the following are true for the local
    \ac{PE}: The \dest{} array is updated and the \source{} array may be safely reused.  
    The values in the \VAR{pSync} array are
    restored to the original values.
}

\apidesctable{ 
    When calling from \Fortran, the \dest{} date types are as follows:
}{Routine}{Data type}
    \apitablerow{shmem\_int8\_and\_to\_all}{Integer, with an element size of 8 bytes.}
    \apitablerow{shmem\_int4\_and\_to\_all}{Integer, with an element size of 4 bytes.}
    \apitablerow{shmem\_comp8\_max\_to\_all}{Complex, with an element size equal to two 8-byte real values.}
    \apitablerow{shmem\_int4\_max\_to\_all}{Integer, with an element size of 4 bytes.}
    \apitablerow{shmem\_int8\_max\_to\_all}{Integer, with an element size of 8 bytes.}
    \apitablerow{shmem\_real4\_max\_to\_all}{Real, with an element size of 4 bytes.}
    \apitablerow{shmem\_real16\_max\_to\_all}{Real, with an element size of 16 bytes.}
    \apitablerow{shmem\_int4\_min\_to\_all}{Integer, with an element size of 4 bytes.}
    \apitablerow{shmem\_int8\_min\_to\_all}{Integer, with an element size of 8 bytes.}
    \apitablerow{shmem\_real4\_min\_to\_all}{Real, with an element size of 4 bytes.}
    \apitablerow{shmem\_real8\_min\_to\_all}{Real, with an element size of 8 bytes.}
    \apitablerow{shmem\_real16\_min\_to\_all}{Real,with an element size of 16 bytes.}
    \apitablerow{shmem\_comp4\_sum\_to\_all}{Complex, with an element size equal to two 4-byte real values.}
    \apitablerow{shmem\_comp8\_sum\_to\_all}{Complex, with an element size equal to two 8-byte real values.}
    \apitablerow{shmem\_int4\_sum\_to\_all}{Integer, with an element size of 4 bytes.}
    \apitablerow{shmem\_int8\_sum\_to\_all}{Integer, with an element size of 8 bytes..}
    \apitablerow{shmem\_real4\_sum\_to\_all}{Real, with an element size of 4 bytes.}
    \apitablerow{shmem\_real8\_sum\_to\_all}{Real, with an element size of 8 bytes.}
    \apitablerow{shmem\_real16\_sum\_to\_all}{Real, with an element size of 16 bytes.}
    \apitablerow{shmem\_comp4\_prod\_to\_all}{ Complex, with an element size equal to two 4-byte real values. }		 
    \apitablerow{shmem\_comp8\_prod\_to\_all}{ Complex, with an element size equal to two 8-byte real values.}
    \apitablerow{shmem\_int4\_prod\_to\_all}{Integer, with an element size of 4 bytes.}
    \apitablerow{shmem\_int8\_prod\_to\_all}{Integer, with an element size of 8 bytes.}
    \apitablerow{shmem\_real4\_prod\_to\_all}{Real, with an element size of 4 bytes.}
    \apitablerow{shmem\_real8\_prod\_to\_all}{Real, with an element size of 8 bytes.}
    \apitablerow{shmem\_real16\_prod\_to\_all}{Real, with an element size of 16 bytes.}
    \apitablerow{shmem\_int8\_or\_to\_all}{Integer, with an element size of 8 bytes.}
    \apitablerow{shmem\_int4\_or\_to\_all}{Integer, with an element size of 4 bytes.}
    \apitablerow{shmem\_int8\_xor\_to\_all}{Integer, with an element size of 8 bytes.}
    \apitablerow{shmem\_int4\_xor\_to\_all}{Integer, with an element size of 4 bytes.}

\apireturnvalues{
    None.
}

\apinotes{  
    All \openshmem{} reduction routines reset the values in \VAR{pSync} before they
    return, so a particular \VAR{pSync} buffer need only be initialized the first
    time it is used. The user must ensure that the \VAR{pSync} array is not being updated on any \ac{PE}
    in the \activeset{} while any of the \acp{PE} participate in processing of an
    \openshmem{} reduction routine. Be careful to avoid the following situations: If
    the \VAR{pSync} array is initialized at run time, some type of synchronization
    is needed to ensure that all \acp{PE} in the working set have initialized
    \VAR{pSync} before any of them enter an \openshmem routine called with the
    \VAR{pSync} synchronization array. A \VAR{pSync} or \VAR{pWrk} array can be
    reused in a subsequent reduction routine call only if none of the \acp{PE} in
    the \activeset{} are still processing a prior reduction routine call that used
    the same \VAR{pSync} or \VAR{pWrk} arrays. In general, this can be assured only
    by doing some type of synchronization. 
}

\begin{apiexamples}

\apifexample
    {This \Fortran reduction example statically initializes the \VAR{pSync} array
    and finds the logical \OPR{AND} of the integer variable \VAR{FOO} across all
    even \acp{PE}.}
    {./example_code/shmem_and_example.f90}
    {}
    
\apifexample
    {This \Fortran example statically initializes the \VAR{pSync} array and finds
    the \OPR{maximum} value of real variable \VAR{FOO} across all even \acp{PE}.}
    {./example_code/shmem_max_example.f90}
    {}

\apifexample
    { This \Fortran example statically initializes the \VAR{pSync} array and finds
    the \OPR{minimum} value of real variable \VAR{FOO} across all the even
    \acp{PE}.}
    {./example_code/shmem_min_example.f90}
    {}

\apifexample
    {This \Fortran example statically initializes the \VAR{pSync} array and finds
    the \OPR{sum} of the real variable \VAR{FOO} across all even \acp{PE}.}
    {./example_code/shmem_sum_example.f90}
    {}

\apifexample
    {This \Fortran example statically initializes the \VAR{pSync} array and finds
    the \OPR{product} of the real variable \VAR{FOO} across all the even \acp{PE}.}
    {./example_code/shmem_prod_example.f90}
    {}

\apifexample
    {This \Fortran example statically initializes the \VAR{pSync} array and finds
    the logical \OPR{OR} of the integer variable \VAR{FOO} across all even
    \acp{PE}.}
    {./example_code/shmem_or_example.f90}
    {}

\apifexample
    {This \Fortran example statically initializes the \VAR{pSync} array and
    computes the exclusive \OPR{XOR} of variable \VAR{FOO} across all even
    \acp{PE}.}
    {./example_code/shmem_xor_example.f90}
    {} 

\end{apiexamples}

\end{apidefinition}
