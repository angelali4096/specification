\apisummary{
    Registers the arrival of a \ac{PE} at a barrier and blocks the \ac{PE}
    until all other \acp{PE} arrive at the barrier and all local
    updates and remote memory updates on the default context are completed.
}

\begin{apidefinition}

\begin{Csynopsis}
void @\FuncDecl{shmem\_barrier\_all}@(void);
\end{Csynopsis}

\begin{apiarguments}

    \apiargument{None.}{}{}

\end{apiarguments}

\apidescription{   
    The \FUNC{shmem\_barrier\_all} routine
    is a mechanism for synchronizing all \acp{PE} in the world team at
    once. This routine blocks the calling \ac{PE} until all \acp{PE} have called
    \FUNC{shmem\_barrier\_all}. In a multithreaded \openshmem
    program, only the calling thread is blocked, however,
    it may not be called concurrently by multiple threads in the same \ac{PE}.

    Prior to synchronizing with other \acp{PE}, \FUNC{shmem\_barrier\_all}
    ensures completion of all previously issued memory stores and remote memory
    updates issued on the default context via \openshmem \acp{AMO} and
    \ac{RMA} routine calls such
    as \FUNC{shmem\_int\_add}, \FUNC{shmem\_put32},
    \FUNC{shmem\_put\_nbi}, and \FUNC{shmem\_get\_nbi}.
}

\apireturnvalues{
    None.
}

\apinotes{
    The \FUNC{shmem\_barrier\_all} routine is equivalent to calling
    \FUNC{shmem\_ctx\_quiet} on the default context followed by
    calling \FUNC{shmem\_team\_sync} on the world team.

    The \FUNC{shmem\_barrier\_all} routine can be used to
    portably ensure that memory access operations observe remote updates in the order
    enforced by initiator \acp{PE}.

    Calls to \FUNC{shmem\_ctx\_quiet} can be performed prior
    to calling the barrier routine to ensure completion of operations issued on
    additional contexts.
}

\begin{apiexamples}

\apicexample
    { The following \FUNC{shmem\_barrier\_all} example is for \Cstd[11] programs:}
    {./example_code/shmem_barrierall_example.c}
    {}

\end{apiexamples}

\end{apidefinition}
