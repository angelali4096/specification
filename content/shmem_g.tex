\apisummary{
    Copies one data item from a remote \ac{PE}
}

\begin{apidefinition}

\begin{C11synopsis}
TYPE @\FuncDecl{shmem\_g}@(const TYPE *source, int pe);
TYPE @\FuncDecl{shmem\_g}@(shmem_ctx_t ctx, const TYPE *source, int pe);
\end{C11synopsis}
where \TYPE{} is one of the standard \ac{RMA} types specified by Table \ref{stdrmatypes}.

\begin{Csynopsis}
TYPE @\FuncDecl{shmem\_\FuncParam{TYPENAME}\_g}@(const TYPE *source, int pe);
TYPE @\FuncDecl{shmem\_ctx\_\FuncParam{TYPENAME}\_g}@(shmem_ctx_t ctx, const TYPE *source, int pe);
\end{Csynopsis}
where \TYPE{} is one of the standard \ac{RMA} types and has a corresponding \TYPENAME{} specified by Table \ref{stdrmatypes}.

\begin{apiarguments}
  \apiargument{IN}{ctx}{A context handle specifying the context on which to perform the operation.
    When this argument is not provided, the operation is performed on
    the default context.}
  \apiargument{IN}{source}{Symmetric address of the source data object.
    The type of \source{} should match that implied in the SYNOPSIS section.}
  \apiargument{IN}{pe}{The number of the remote \ac{PE} on which \VAR{source} resides.}
\end{apiarguments}

\apidescription{
  These routines provide a very low latency get capability for single elements
  of most basic types. 
}

\apireturnvalues{
    Returns a single element of type specified in the synopsis.
}

\begin{apiexamples}

\apicexample
    {The following \FUNC{shmem\_g} example is for \Cstd[11] programs:}
    {./example_code/shmem_g_example.c}
    {}
\end{apiexamples}

\end{apidefinition}
