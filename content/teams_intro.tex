The \acp{PE} in an \openshmem program communicate using either
point-to-point routines---such as \ac{RMA} and \ac{AMO} routines---that specify the \ac{PE} number of the target
\ac{PE}, or collective routines that operate over a set of \acp{PE}.
\openshmem teams allow programs to group a set of \acp{PE} for
communication.
Team-based collective operations include all \acp{PE}
in a valid team.
Point-to-point communication can make use of team-relative \ac{PE}
numbering through team-based contexts (see Section~\ref{sec:ctx}) or
\ac{PE} number translation.

\subsubsection*{Predefined and Application-Defined Teams}

An \openshmem team may be predefined (i.e., provided by the \openshmem
library) or defined by the \openshmem application.
An application-defined team is created by ``splitting'' a parent team into
one or more new teams---each with some subset of \acp{PE} of the
parent team---via one of the \FUNC{shmem\_team\_split\_*} routines.

All predefined teams are valid for the duration of the \openshmem
portion of an application.
Any team successfully created by a \FUNC{shmem\_team\_split\_*}
routine is valid until it is destroyed.
All valid teams have a least one member.

\subsubsection*{Team Handles}

A \emph{team handle} is an opaque object with type \CTYPE{shmem\_team\_t}
that is used to reference a team.
Team handles are not remotely accessible objects.
The predefined teams may be accessed via the team handles listed in
Section~\ref{subsec:library_handles}.

\openshmem communication routines that do not accept a team handle
argument operate on the world team, which may be accessed through
the \LibHandleRef{SHMEM\_TEAM\_WORLD} handle.
The world team encompasses the set of all \acp{PE} in the \openshmem
program, and a given \ac{PE}'s number in the world team is equal to the
value returned by \FUNC{shmem\_my\_pe}.

A team handle may be initialized to or assigned the value
\CONST{SHMEM\_TEAM\_INVALID} to indicate that handle does not
reference a valid team.
When managed in this way, applications can use an equality comparison
to test whether a given team handle references a valid team.

\subsubsection*{Thread Safety}

When it is allowed by the threading model provided by the OpenSHMEM
library, a team may be used concurrently in non-collective operations
(e.g., \FUNC{shmem\_team\_my\_pe}) by multiple threads within the
\ac{PE} where it was created.
A team may not be used concurrently by multiple threads in the same \ac{PE} for
collective operations. However, multiple collective operations on different
teams may be performed in parallel.

\subsubsection*{Collective Ordering}

In \openshmem, a team object encapsulates resources used to communicate
between \acp{PE} in collective operations. When calling multiple subsequent
collective operations on a team, the collective operations---along with any
relevant team based resources---are matched across the \acp{PE} in the team
based on ordering of collective routine calls. It is the responsibility
of the user to ensure that team-based collectives occur in the same program order
across all \acp{PE} in a team.

For a full discussion of collective semantics, see Section~\ref{subsec:coll}.

\subsubsection*{Team Creation}

Team creation is a collective operation on the parent team object. New teams
result from a \FUNC{shmem\_team\_split\_*} routine, which takes a parent team
and other arguments and produces new teams that contain a subset of the \acp{PE}
that are members of the parent
team. All \acp{PE} in a parent team must participate in a split operation
to create new teams. If a \ac{PE} from the parent team is not a member of any
resulting new teams, it will receive a value of \CONST{SHMEM\_TEAM\_INVALID}
as the value for the new team handle.

Teams that are created by a \FUNC{shmem\_team\_split\_*} routine may be
provided a configuration argument that specifies attributes of each new team.
This configuration argument is of type \CTYPE{shmem\_team\_config\_t}, which
is detailed further in Section~\ref{subsec:shmem_team_config_t}.

\acp{PE} in a newly created team are consecutively numbered starting with
\ac{PE} number 0. \acp{PE} are ordered by their \ac{PE} number in
the parent team. Team relative \ac{PE}
numbers can be used for point-to-point operations through team-based
contexts (see Section~\ref{sec:ctx}) or using the translation routine
\FUNC{shmem\_team\_translate\_pe}.

Split operations are collective and are subject to the constraints on
team-based collectives specified in Section~\ref{subsec:coll}. In particular,
in multithreaded executions, threads at a given \ac{PE} must not perform
simultaneous split operations on the same parent team.
Team creation operations are matched across participating PEs based
on the order in which they are performed. Thus, team creation events must also
occur in the same order on all \acp{PE} in the parent team.

Upon completion of a team creation operation, the parent and any resulting child teams
will be immediately usable for any team-based operations, including creating new child
teams, without any intervening synchronization.
